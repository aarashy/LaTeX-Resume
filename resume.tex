%-------------------------
% Adapted by Aarash Heydari from Sourabh Bajaj (https://github.com/sb2nov/resume)
%------------------------

\documentclass[letterpaper,11pt]{article}

\usepackage{latexsym}
\usepackage[empty]{fullpage}
\usepackage{titlesec}
\usepackage{marvosym}
\usepackage[usenames,dvipsnames]{color}
\usepackage{verbatim}
\usepackage{enumitem}
\usepackage[pdftex]{hyperref}
\usepackage{fancyhdr}
\usepackage{textcomp}



\pagestyle{fancy}
\fancyhf{} % clear all header and footer fields
\fancyfoot{}
\renewcommand{\headrulewidth}{0pt}
\renewcommand{\footrulewidth}{0pt}

% Adjust margins
\addtolength{\oddsidemargin}{-0.375in}
\addtolength{\evensidemargin}{-0.375in}
\addtolength{\textwidth}{1in}
\addtolength{\topmargin}{-.5in}
\addtolength{\textheight}{1.0in}

\urlstyle{same}

\raggedbottom
\raggedright
\setlength{\tabcolsep}{0in}

% Sections formatting
\titleformat{\section}{
  \vspace{-4pt}\scshape\raggedright\large
}{}{0em}{}[\color{black}\titlerule \vspace{-5pt}]

%-------------------------
% Custom commands
\newcommand{\resumeItem}[2]{
  \item\small{
    \textbf{#1}{: #2 \vspace{-2pt}}
  }
}

\newcommand{\resumeSubheading}[4]{
  \vspace{-1pt}
    \begin{tabular*}{0.97\textwidth}{l@{\extracolsep{\fill}}r}
      \textbf{#1} & #2 \\
      \textit{\small#3} & \textit{\small #4} \\
    \end{tabular*}\vspace{-5pt}
}

\newcommand{\resumeSubItem}[2]{\resumeItem{#1}{#2}\vspace{-4pt}}

\renewcommand{\labelitemii}{$\circ$}

\newcommand{\resumeSubHeadingListStart}{\begin{itemize}[leftmargin=*]}
\newcommand{\resumeSubHeadingListEnd}{\end{itemize}}
\newcommand{\resumeItemListStart}{\begin{itemize}}
\newcommand{\resumeItemListEnd}{\end{itemize}\vspace{-5pt}}

%-------------------------------------------
%%%%%%  CV STARTS HERE  %%%%%%%%%%%%%%%%%%%%%%%%%%%%


\begin{document}

%----------HEADING-----------------
\begin{tabular*}{\textwidth}{l@{\extracolsep{\fill}}r}
  \textbf{\href{http://linkedin.com/in/aarashy}{\Large Aarash Heydari}} & Email : \href{mailto:aheyd@berkeley.edu}{aheyd@berkeley.edu}\\
  \href{http://linkedin.com/in/aarashy}{http://linkedin.com/in/aarashy} & Mobile : +1-540-282-8104
\end{tabular*}

\section{Statement of Purpose}
I am an engineer who takes the time to understand things deeply but prefers to execute quickly. My experiences working at early- to mid-stage startups gave me the opportunity to wear many hats and perform functions that stretched beyond the role of an architect. I love writing code and explaining complicated concepts in computing. Above all, I am versatile. \newline

In my next role, I hope to lead business-critical functions and be part of a team that challenges me to deepen my technical abilities and soft skills.


%-----------EDUCATION-----------------
\section{Education}
    \resumeSubheading
      {University of California, Berkeley}{Berkeley, CA}
      {Bachelor of Arts: Computer Science;  GPA: 3.73}{Aug. 2015 -- May 2019}
      \resumeItemListStart
        \resumeItem{TA / Student Instructor}
          {Taught four classes over my last five semesters:\\
Data Structures \quad\textbar\quad Artificial Intelligence\quad\textbar\quad Algorithms (two semesters)\quad\textbar\quad Machine Learning}
      \resumeItemListEnd

    \vspace{2mm}

%-----------EXPERIENCE-----------------
\section{Experience}
    \resumeSubheading
      {Dataland (Y Combinator, S20)}{New York, NY}
      {Founding Engineer}{August 2021 - Present}
      
      \vspace{4mm}
      Dataland is a private startup that helps internal teams use one UI to work across all their systems of record. As the first hire, I built their sophisticated product from the ground up, gained expertise in high-performance systems programming in Rust, and developed very strong engineering fundamentals.
      
      \resumeItemListStart
      \resumeItem{SQL time machine}
          {Built a SQL CLI application with extended time travel syntax, i.e. ``time travel monday 4pm". Followup queries are executed from the perspective of that point in time. This is made possible by replicating a PostgreSQL transaction log over Kafka to construct a history database that includes every version of every row, and utilizing the open-source \href{https://github.com/pganalyze/libpg_query}{\textbf{\underline{PostgreSQL query parser}}} to rewrite queries to incorporate time travel CTEs.}
      \resumeItem{Kubernetes resource controllers}
          {Developed an infrastructure-as-code deployment experience for customers to submit table schemas and JavaScript trigger functions via the Dataland CLI. Used Kubernetes operators to allocate and monitor DB resources, do SQL migrations, and run user-provided code in sandboxed V8 isolates (Deno).}
        \resumeItem{DevOps and infrastructure}
          {Operationally managed backend microservices in GKE. This included deploying production upgrades via kpt, configuring Kubernetes node resources and workload requests based on application needs, and using taints and tolerations to pin special workloads. Used Cloud DNS to configure static IP masquerading. Used GCP logging and alerting to monitor system health.}
      \resumeItem{Open-source}
          {Utilized and deeply studied a variety of open source projects to understand their architecture. To name a few, In the C programming language, I have read thousands of lines of \href{https://github.com/postgres/postgres}{\textbf{\underline{PostgreSQL}}} and \href{https://github.com/confluentinc/librdkafka}{\textbf{\underline{the Kafka client library}}}. In Go, I studied and utilized \href{https://github.com/ory/kratos}{\textbf{\underline{Ory Kratos}}} to build a simple and secure Authn/Authz service. In C++, I have contributed bug reports, issues and pull requests to \href{https://github.com/duckdb/duckdb/issues?q=author\%3Aaarashy}{\underline{\textbf{DuckDB}}}, an OLAP SQL engine. In Rust, I have contributed to the \href{https://github.com/bikeshedder/deadpool}{\textbf{\underline{Postgres connection pool}}} crate and the \href{https://github.com/apache/arrow-rs}{\textbf{\underline{Apache Arrow}}} crate.}
    \resumeItemListEnd
    \vspace{5mm}
    \resumeSubheading
      {Clumio}{Santa Clara, CA}
      {Software Engineer / Tech Lead}{July 2019 - July 2021}
      
      \vspace{4mm}
      Clumio is a private startup that provides cloud data protection at scale. As a cross-functional member and lead of multiple teams, I architected common platform components, delivered features that enhanced customer UX, and was responsible for the reliability of critical infrastructure. 
      
      \resumeItemListStart
        \resumeItem{Alerts and progress logs}
          {As tech lead, spearheaded the overhaul of the internal and external alerting framework. Identified a lack of transparency into the step-by-step progression of backend workflows. Designed and delivered alerting and log solutions that delight customers. Educated fellow developers on how to integrate with this. Mentored a Customer Success engineer, empowering him to build his first microservice which proactively opens Zendesk tickets in response to failures in the system, driving high customer confidence in the platform.}
        \resumeItem{Public REST APIs/SDKs}
          {As tech lead, managed the platform, design, deployment, and versioning of 150+ APIs built on Swagger/OpenAPI and API Gateway. Met weekly with developers and PMs, working across time zones to review APIs and documentation of all engineering teams. Learned about every feature by way of reviewing all APIs. \newline \href{https://clumio.com/an-outstanding-user-experience-for-api-developers-think-ux-for-rest-apis/}{\underline{\textbf{Wrote a blog about Clumio's API design principles and architecture.}}}}
      \resumeItem{Scheduler team}
          {Scheduled 100k backup jobs for EC2, RDS, and VM disks per day using SNS work queues. Managed the health of key platform services, including a configurable job scheduler, resource manager, and task manager. Participated in on-call rotation to handle escalations.}
      \resumeItem{BI dashboards}
          {Queried against a 40 million row datalake by integrating Snowflake with embedded Sigma dashboards, powering unprecedented visibility into customer AWS compute and cost footprints. Iterated with PMs to ensure  we were presenting the most useful data in the most digestible way.}
      \resumeItemListEnd
    \vspace{5mm}
    \resumeSubheading
      {Microsoft, Azure Identity}{Redmond, WA}
      {Software Engineer Intern}{May 2018 - Aug. 2018}
          \vspace{2mm}
      \resumeItemListStart
        \resumeItem{Feature flag auto-rollout}
          {Used Azure Functions to automate the configuration management infrastructure for a global LDAP service to minimize human error in deployment, saving {\raise.17ex\hbox{$\scriptstyle\sim$}}3 hours of engineer time per enabled feature.}
      \resumeItemListEnd
    \vspace{5mm}
    \resumeSubheading
      {Datalogue (Acquired by Nike)}{Montreal, Quebec}
      {Backend Software Engineer Intern}{May 2017 - Aug. 2017}
      
          \vspace{4mm}
Datalogue used ML to prepare, integrate, and transform data from a variety of sources across an enterprise. 
      \resumeItemListStart
        \resumeItem{Data pipeline}
      {Expanded the functionality of Scala data pipelines that deliver massive processing power by transforming data formats including JSON, Excel, XML, and MongoDB to and from a proprietary Graph format.}
      \resumeItemListEnd

\section{Research Projects}
    \resumeSubheading
      {Big Data in Radiology (BDRad)}{UCSF Radiology and Biomedical Imaging Department}
      {Advised by Dr. Jae Ho Sohn}{Sept. 2018 - December 2019}
      
      \vspace{4mm}
      \href{https://radiology.ucsf.edu/research/labs/physics/big-data}{\textbf{\underline{BDRad}}} is an interdisciplinary research group which brings together data scientists, computer scientists, and radiologists in order to understand how big data computational resources and computer vision advances can provide innovative solutions in imaging sciences.
      
  \resumeSubHeadingListStart
    \resumeSubItem{Annotation/visualization system for 3D cross-sectional images}
      {Integrated a Python plugin for a medical image viewer app to translate between segmentation maps and tabular point-and-radius CSV data. Used this to visualize ML predictions on CT scans and provide radiologists a highly efficient annotation system for producing training data.}
    \resumeSubItem{Presentation at Radiological Society of North America, 2019}{``A DICOM-Embedded Annotation System for 3D Cross-Sectional Imaging Data" (\href{https://github.com/bdrad/pyOsirixScripts/blob/master/rsna2019_poster_aarash.pdf}{\underline{\textbf{Poster}}} \textbar \hspace{0.5mm} \href{https://github.com/aarashy/dicom-viewer}{\underline{\textbf{Source Code}}}).}
    \resumeSubItem{Peer-reviewed publication} {``High precision localization of pulmonary nodules on chest CT utilizing axial slice number labels" (\href{https://link.springer.com/article/10.1186/s12880-021-00594-4}{\underline{\textbf{Publication}}})}
    \resumeSubItem{Weekly reading group}
      {Participated in weekly discussions of the group's projects and machine learning theory. Presented  lectures on topics such as \href{https://docs.google.com/presentation/d/1bq07GSe7ilcrZInrKemy1EVERqXM25jV_axE1Uhba-c/edit?usp=sharing}{\textbf{\underline{ROC curves and confidence intervals}}}.}
    \resumeSubHeadingListEnd
    
    \vspace{5mm}
    
    \resumeSubheading
      {Fake News and Misinformation}{UC Berkeley}
      {Advised by Professor Gireeja Ranade}{Feb. 2018 - July 2019}
      
    \vspace{4mm}
      Professor Ranade assembled a research group to perform data-driven studies on the sociological consequences of social media on news and politics.
      
  \resumeSubHeadingListStart
    \resumeSubItem{Data collection}
      {Used YouTube APIs to collect and analyze data on comments in political media coverage.}
    \resumeSubItem{ML} {Used scikit-learn to apply Random Forest, SVM, etc. to predict the bias category of YouTube videos based on comment statistics.}
    \resumeSubItem{Publication} {``YouTube Chatter: Understanding Online Comments Discourse on Misinformative and Political YouTube Videos", First Author (\href{https://arxiv.org/abs/1907.00435}{\underline{\textbf{Publication}}} \textbar \hspace{0.5mm} \href{https://github.com/aarashy/YouTubeComments}{\underline{\textbf{Source Code}}})}
    \resumeSubHeadingListEnd

    \vspace{4mm}

%--------PROGRAMMING SKILLS------------
\section{Skills}
 \resumeSubHeadingListStart
    \resumeSubItem{Languages}{Rust, Python, Go, Scala, Java, TypeScript, C, C++}
    \resumeSubItem{Technologies}{AWS, GCP, Kafka, Snowflake, PostgreSQL,  Unix, Docker, Kubernetes (deep mastery)}
    \resumeSubItem{Humanities}{Eagle Scout,  Farsi, French, Classical Piano, Jazz Drums}
 \resumeSubHeadingListEnd


%-------------------------------------------
\end{document}
